\documentclass[12pt]{article}

\usepackage{answers}
\usepackage{setspace}
\usepackage{graphicx}
\usepackage{enumitem}
\usepackage{multicol}
\usepackage{mathrsfs}
\usepackage[margin=1in]{geometry} 
\usepackage{amsmath,amsthm,amssymb}
\usepackage[ngerman]{babel}

\newcommand{\N}{\mathbb{N}}
\newcommand{\Z}{\mathbb{Z}}
\newcommand{\C}{\mathbb{C}}
\newcommand{\R}{\mathbb{R}}

\DeclareMathOperator{\sech}{sech}
\DeclareMathOperator{\csch}{csch}

\newenvironment{theorem}[2][Theorem]{\begin{trivlist}
		\item[\hskip \labelsep {\bfseries #1}\hskip \labelsep {\bfseries #2.}]}{\end{trivlist}}
\newenvironment{definition}[2][Definition]{\begin{trivlist}
		\item[\hskip \labelsep {\bfseries #1}\hskip \labelsep {\bfseries #2.}]}{\end{trivlist}}
\newenvironment{proposition}[2][Proposition]{\begin{trivlist}
		\item[\hskip \labelsep {\bfseries #1}\hskip \labelsep {\bfseries #2.}]}{\end{trivlist}}
\newenvironment{lemma}[2][Lemma]{\begin{trivlist}
		\item[\hskip \labelsep {\bfseries #1}\hskip \labelsep {\bfseries #2.}]}{\end{trivlist}}
\newenvironment{exercise}[2][Exercise]{\begin{trivlist}
		\item[\hskip \labelsep {\bfseries #1}\hskip \labelsep {\bfseries #2.}]}{\end{trivlist}}
\newenvironment{solution}[2][Solution]{\begin{trivlist}
		\item[\hskip \labelsep {\bfseries #1}]}{\end{trivlist}}
\newenvironment{problem}[2][Problem]{\begin{trivlist}
		\item[\hskip \labelsep {\bfseries #1}\hskip \labelsep {\bfseries #2.}]}{\end{trivlist}}
\newenvironment{question}[2][Question]{\begin{trivlist}
		\item[\hskip \labelsep {\bfseries #1}\hskip \labelsep {\bfseries #2.}]}{\end{trivlist}}
\newenvironment{corollary}[2][Corollary]{\begin{trivlist}
		\item[\hskip \labelsep {\bfseries #1}\hskip \labelsep {\bfseries #2.}]}{\end{trivlist}}

\begin{document}
	\title{Interaktive Computergrafik}
	\author{Michael Gabler}
	\maketitle
	\tableofcontents
	\newpage

	\section{Grundlagen}
	Computergrafik beschreibt das Erstellen von 2D-Bildern aufgrund von 3D-Daten.\\
	\textbf{Anwendungsgebiete}
	\begin{itemize}
		\item Human-Computer-Interaction
		\item CAD \& (wissenschaftliche) Visualisierung
		\item Filme
		\item Computer Spiele
	\end{itemize}
	\textbf{3D-Repräsentation} Wie können Objekte als 3D-Modell abgebildet werden?
	\begin{itemize}
		\item Implizite Parameter (z.B. als Funktion)
		\item Oberfläche annähernd beschrieben durch Dreiecke oder Polygone (manuell, Laser Scanner, Fotos von allen Seiten)
		\item Volume Solids (z.B. durch Sensoren wie MRT oder CT)
	\end{itemize}
	\textbf{Animation} z.B. über Referenzpunkte, die mit echter Welt gemappt werden\\
	\textbf{Rendering} Abbilden von 3D-Daten auf 2D-Repräsentation z.B. durch Raytracing oder Rasterization\\
	\textbf{Immersion} Maß in wie weit eine virtuelle Darbietung äußere, reale Wahrnehmungen ausgrenzt und diese durch virtuelle ersetzt.\\
	\textbf{Präsenz/Presence} In wie weit fühlt sich ein Subjekt in einer Umgebung angekommen/eingebungen auch wenn es sich in einer anderen befindet.\\
	\textbf{Digitalisierung analoger Signale}\\
	\includegraphics[width=\linewidth]{figures/digitalisierung.png}\\
	\textbf{Rastergrafik} Grafik wird als Pixel beschrieben, die jeweils eine Farbe haben $\rightarrow$ Skalierung schwierig. Beispiel: JPG, PNG, GIF, TIFF, PBM\\
	\textbf{Vektorgrafik} Inhalt der Grafik wird durch geometrische Formen beschrieben. Kann gerastert und beliebig skaliert werden. Beispiel: SVG, PS (Postscript), CGM, IGES, DWF/DXF\\


	\subsection{Menschliche Wahrnehmung von Licht}
	zwischen 380nm (violet/blau) und 780nm (rot)\\
	\textbf{Zapfen/Cones} Farbliche Wahrnehmung (ca. 6 Millionen) je für einen Farbkanal zuständig (64 \% rot, 32 \% grün, 4 \% blau)\\
	\textbf{Stäbchen/Rods} Helligkeitswahrnehmung (ca. 120 Millionen)\\
	\textbf{Farbsysteme} Repräsentation durch unterschiedliche Modelle, wie:
	\begin{itemize}
		\item biologisch orientiert: CIE XYZ
		\item Hardware-orientiert: RGB, CMY, CMYK (mit Schwarz, um Tinte zu sparen)
		\item Anwender-orientiert: HSV, HSB
	\end{itemize}
	\textbf{Steven's Power Law} physikalische Intensität (Helligkeit) ist nicht proportional zur wahrnehmbaren Helligkeit.\\
	\includegraphics[width=\linewidth]{figures/stevens-law.png}\\
	\textbf{Gamma Korrektur} korrigiert physikalische Intensität, um kontinuierlichen wahrnehmbaren Intensitätszuwachs zu bekommen.\\
	\includegraphics[width=\linewidth]{figures/gamma-correction.png}
	$$n = \lfloor I^{\frac{1}{\gamma}} 2^N \rfloor$$
	mit $I \in [0,1]$, $n \in [0,2^N]$: Abbildung der physikalischen Intensität auf wahrnehmungskorrigierte mit $N$ Bit Genauigkeit.

	\subsection{Rasterisierung von Vektorgrafiken}
	\textbf{Digital Differential Analyzer (DDA)} Rastern von Linien zwischen zwei beliebigen Punkten $p_1$ und $p_2$. Linie kann als Funktion $y = mx + b$ repräsentiert werden.\\
	\includegraphics[width=\linewidth]{figures/dda.png}\\
	\textbf{Bresenham's Algorithmus} Verfeinerung von DDA\\
	\includegraphics[width=\linewidth]{figures/bresenham.png}\\
	\textbf{Aliasing} Da Pixel entweder an oder aus sind (haben Farbe oder nicht), bildet sich eine Treppe beim Rastern von Linien. Kann beim Samplen auftreten\\
	$\rightarrow$ \textbf{Nyquist-Shannon Sampling Theorem} Sample Frequenz $>=$ Doppelte Signal Frequenz\\
	$\Rightarrow$ \textbf{Antialiasing} hat keine harten Übergänge sondern bildet "Farbverlauf" am Kantenrand (Pixel sind an, aus oder abgeschwächt farbig) durch Supersampling (mehr Punkte als Raster berechnen) und Durchschnittsbildung\\
	\textbf{Supersampling} feinere Auflösung als Zielbild wählen und regelmäßig samplen (beste Ergebnisse), zufällige Punkte wählen im gesamten Bild, zufällige Punkte in definierten Räumen

	\subsection{Vektoren}
	\textbf{Skalarprodukt/Dot-Produkt} $u \cdot v = (u_0, u_1, u_2)^T \cdot (v_0, v_1, v_2)^T = u_0 v_0 + u_1 v_1 + u_2 v_2$\\
	\textbf{Vektorlänge} $||v|| = \sqrt{v \cdot v}$\\
	\textbf{Kreuxprodukt} $w = u \times v = 
		\begin{bmatrix}
			u_1 v_2 - u_2 v_1\\
			u_2 v_0 - u_0 v_2\\
			u_0 v_1 - u_1 v_0
		\end{bmatrix}$\\
	$w$ ist orthogonal zu $u$ und $v$. Sie bilden ein rechthand-(Koordinaten)-System

	\subsection{Homogene Koordinaten}
	Repräsentieren von Vektoren als homogen, um Transformationen durchführen zu können. (wie in Robotik 1). Vierte Komponente ist 0 für Vektoren, 1 für Punkte.\\
	\includegraphics[width=\linewidth]{figures/homogene-koordinaten.png}
	\paragraph{Fundamental rotation matrices} $R_{n}(\phi)$ describes the rotation around the n axes with the angle $\phi$.\\
	\begin{equation}
	R_{1}(\phi) = 
	\begin{bmatrix}
	1 & 0 & 0\\
	0 & \cos \phi & -\sin \phi\\
	0 & \sin \phi & \cos \phi
	\end{bmatrix}
	\end{equation}
	\begin{equation}
	R_{2}(\phi) = 
	\begin{bmatrix}
	\cos \phi & 0 & \sin \phi\\
	0 & 1 & 0\\
	-\sin \phi & 0 & \cos \phi
	\end{bmatrix}
	\end{equation}
	\begin{equation}
	R_{3}(\phi) = 
	\begin{bmatrix}
	\cos \phi & -\sin \phi & 0\\
	\sin \phi & \cos \phi & 0\\
	0 & 0 & 1
	\end{bmatrix}
	\end{equation}
	
	\paragraph{Homogenous transformation} Describe coordinates as homogenous coordinates to describe rotations and translations\\
	vector $q$ as homogenous coordinates is $[q_{1} q_{2} q_{3} 1]^T$\\
	Homogenous transformation matrix
	\begin{equation}
	T = 
	\begin{bmatrix}
	R & p\\
	\eta^T & \sigma
	\end{bmatrix}
	\end{equation}
	with\\
	$R \in \R^{3 \times 3}$ is a rotation matrix\\
	$p \in \R^{3 \times 1}$ is a translation vector\\
	$\sigma \in \R$ is a scaling factor (usually 1)\\
	$\eta^T \in \R^{1 \times 3}$ is a perspective vector, here zero vector	\begin{equation}
	Rot(\phi, k) = 
	\begin{bmatrix}
	& & & 0\\
	& R_{k}(\phi) & & 0\\
	& & & 0\\
	0 & 0 & 0 & 1
	\end{bmatrix}, 
	Tran(p) = 
	\begin{bmatrix}
	1 & 0 & 0 & p_{1}\\
	0 & 1 & 0 & p_{2}\\
	0 & 0 & 1 & p_{3}\\
	0 & 0 & 0 & 1
	\end{bmatrix}
	\end{equation}

	\paragraph{Reflection} Use identity matrix with $-1$ for the reflection axis. Reflection at a point is reflection at the intersection of three orthogonal planes ($-1$ for every axis).

	\paragraph{Transformation for abitraty points} Translate everything so that the transformation point is located at the origin. Do the Transformation. Translate everything back.
	
	\paragraph{Inverse homogenous transformation} If the transformation matrix $T$ maps coordinates from coordinate frame $A$ to $B$, the inverse $T^{-1}$ maps coodinates from $B$ to $A$.
	\begin{equation}
	T^{-1} = 
	\begin{bmatrix}
	& R^T & & -R^T p\\
	0 & 0 & 0 & 1
	\end{bmatrix}
	\end{equation}
	with $\eta = 0$ and $\sigma = 1$
	
	\section{Raytracing}
	Rendern von 3D-Objekten als 2D-Repräsentation. Orientiert an Physik, dadurch sehr realistische Darstellungen möglich. Strahlen werden von Kamera in die Szene gesendet. Schneiden diese ein 3D-Objekt, wird der Strahl von dort zu einer Lichtquelle verfolgt und der aussendende Pixel entsprechend gefärbt.\\
	\begin{verbatim}
		raytrace(scene, camera, image):
		   # For all pixels in image
		   for (x, y) in image:
		      # 1. Generate ray through pixel
		      ray = camera.generateRay(x, y)
		      # 2. Find closest intersection with scene
		      hit = scene.intersect(ray)
		      # 3. Calculate light intensity
		      color = shade(hit, scene)
		      # 4. Set pixel color
		      image.set(x, y, color)		
	\end{verbatim}
	\includegraphics[width=\linewidth]{figures/raytracing.png}\\
	\textbf{Photon} Lichtstrahl mit bestimmter Energie bzw. Farbe
	$$E = h \cdot f$$
	mit $E$: Energie, $h$: Planksche Konstante, $f$: Frequenz
	$$1 Lumen = 4 \cdot 10^{15} Photonen/sec$$
	\textbf{Absorption} Photon verschwindet, wird von Gegenstand geschluckt\\
	\textbf{Reflektion} Photon prallt an Oberfläche ab\\
	\textbf{Refraktion} Photon geht durch eine Oberfläche hindurch (z.B. Glas)\\
	\textbf{Ray/Strahl} Dargestellt als Startpunkt mit Richtungsvektor: $x(t) = x_0 + t \vec{d}$
	\textbf{Licht-Material Interaktion} Mögliche Modelle zur realistischen Farbermittlung:
	\begin{itemize}
		\item Quantum Theorie (Emission, Absorption)
		\item Spezielle Relativität (aberration, blueshift, redshift, time dilatation)
		\item Wellenoptik (diffraction, dispersion, Interferenz)
		\item Geometische Optik (Reflektion, Refraktion)
	\end{itemize}

	\subsection{Shading}
	Verwende Objektfarbe bei Rayintersection. Naiver Ansatz, da physikalische Gesetze wie Reflektion oder Schatten und Objektstruktur ignoriert werden. Gleichmäßig gefärbte Objekte.

	\subsection{Phong Relektionsmodell}
	Modell zur Farbberechnung. Basiert auf geometischer Optik. Setzt sich zusammen aus der Farbe (Ambient) + Oberflächenbeschaffenheit und Schatten (Diffuse) + Reflektion (Specular). Verwendet folgende Vereinfachungen:
	\begin{itemize}
		\item Oberfläche: Isotropic, betrachtet nicht die Wellenlänge und Polarisierung
		\item Raum: nimmt Vacuum an, keine atmospherischen Effekte
		\item Licht: Lichtquellen als einzelne Punkte
	\end{itemize}
	\textbf{Ambient}\\
	\includegraphics[width=\linewidth]{figures/phong-ambient.png}\\
	\textbf{Diffuse} Betrachtet reflektiertes Licht von Oberflächen (unterschiedlich je nach Oberfläche matt/glänzend)\\
	\includegraphics[width=\linewidth]{figures/phong-diffuse.png}
	$$L_d(p) = k_d \sum_j L_j \max(0, n \cdot s_j)$$
	mit $L_d$: Reflektiertes Licht (Diffuse-Anteil)\\
	$k_d$: Material-Koeffizient für Diffuse Reflektion\\
	$L_j$: Licht (Farbe) der Lichtquelle $j$\\
	$n$: Normale vom Punkt $p$ zur Oberfläche\\
	$s_j$: Vektor von $p$ zur Lichtquelle $j$\\
	\textbf{Specular} Betrachtet starke Reflektionen für glänzende Oberflächen (weiße/helle Punkte)\\
	\includegraphics[width=\linewidth]{figures/phong-specular.png}\\
	\textbf{Reflektionsberechnung} Berechnung des Reflektionsvektors\\
	\includegraphics[width=\linewidth]{figures/reflektionsvektor.png}

	\section{Scene Graph}
	Ein Szenengraph repräsentiert alle Objekte einer Szene als Graph. Dabei werden Positionen immer relativ zum Elternelement angegeben. Wird eine globale Position relativ zum Weltkoordinatensystem benötigt, müssen alle Transformationen entlang des Graphs vom Ursprung zum Objekt angewendet werden.\\
	\includegraphics[width=\linewidth]{figures/scene-graph.png}

	\subsection{Koordinatensysteme}
	Man unterscheidet mehrere Koordinatensysteme\\
	\textbf{Weltkoordinatensystem $\mathbf{W}$} Globales Referenzsystem für alle Objekte\\
	\textbf{Objektsystem $\mathbf{O}$} Relatives System für jedes Objekt. Urspung im Objektursprung. Verwendet z.B. um Punkte eines Objekts relativ zum Ursprung anzugeben.\\
	\textbf{Kamerasystem $\mathbf{C}$} System relativ zur Kameraposition

	\subsection{Rendering}
	Zum Rendern des Scene-Graphs wird das Visitor-Pattern verwendet.\\
	\includegraphics[width=\linewidth]{figures/visitor-pattern.png}

	\section{Rasterisierung}

	% TODO weiter mit 09-rasterization-pipeline

	% \section{Grundlagen}
	% Neuronales Netzwerk ist Funktion, die auf Eingabedaten angewendet wird.\\
	% \textbf{Optimierung} durch Minimierung der Loss-Funktion\\
	% \textbf{Loss-Funktion} Maß, wie gut das Netzwerk Vorhersagen trifft. Berechnet sich aus Vorhersage und tatsächlichen Werten (Ground Truth).
	% \begin{itemize}
	% 	\item Euklidischer Loss, Mean-Squared-Error: $l_2 = \frac{1}{2N} \sum_i (f_\theta(x_i)-t_i)^2$
	% 	\item Negative-Log-Likelihood, Cross-Entropy: $NLL = -\frac{1}{|D|}\sum_i \log[f_\theta(x_i)|_{t_i}]$
	% \end{itemize}
	% \textbf{Konfusionsmatrix} welche Klassen werden wie oft mit welcher Klasse verwechselt?\\
	% \textbf{Linearisiertes Speichern mehrdimensionaler Objekte}\\
	% \includegraphics[width=0.25\linewidth]{figures/row-column-major.png}\\

	
	% \subsection{Training}
	% Daten werden aufgeteilt in Train/Validierung/Test (z.B. 60/20/20)\\
	% \textbf{Datenaugmentierung} Generieren zusätzlicher Daten (z.B. bei Bildern) durch Spiegelung, Rotation, Skalierung, Anpassung Helligkeit und Farbe, etc.\\
	% \textbf{Epoche} Verarbeitung aller Trainingsdaten\\
	% \textbf{Iteration} Verarbeitung eines Batches\\
	% \textbf{Batch} Mehrere Trainingsbeispiele werden gerechnet bevor Gewichte einmal geupdated werden (z.B. 10 Beispiele pro Batch)\\
	% \textbf{Learning Rate} Faktor $\eta$, wie stark das Netzwerk durch die Deltas verändert werden soll (d.h. wie schnell es lernt bzw. seine Meinung ändert). Wird beim Update der Gewichte verwendet.
	% \textbf{Evaluation auf Validierungsdaten} zur Anpassung der Hyperparameter (Learning-Rate, Netzstruktur, ...)\\
	% \textbf{Evaluation auf Testdaten} einmalig, um Genauigkeit des trainierten Netzes zu ermitteln
	% \textbf{Forward-Pass} Berechnen des Outputs des Netzwerkes für bestimmte Eingabedaten (z.B. ein Batch)\\
	% \textbf{Backward-Pass} Bilden der partiellen Ableitung für jeden Input in jedem Layer und Speichern der Werte als Deltas\\
	% \includegraphics[width=\linewidth]{figures/backpropagation.png}\\
	% \textbf{Berechnung der Gewicht-Deltas} Bilden der partiellen Ableitung für jedes Gewicht jedes Layers und Speichern der Werte als Deltas. Zur Berechnung sind die Deltas der Outputs (siehe Backward-Pass) erforderlich. Für Batches werden die Deltas der Gewichte aufsummiert und nach dem Batch geupdated.\\
	% \includegraphics[width=\linewidth]{figures/calculate-delta-weights.png}\\
	% \textbf{Update der Gewichte} Die Gewichte werden nun geupdated, in dem die Deltas der Gewichte mit den ursprünglichen Gewichten verrechnet werden. Dafür gibt es verschiedene Optimierungsverfahren:
	% \begin{itemize}
	% 	\item Gradient Descent: für Gewicht $w'_{ij} = w_{ij} - \eta \cdot \delta w_{ij}$
	% 	\item Adam-Optimizer: robuster gegeüber schlecht gewählter Learning-Rate
	% 	\item Adagrad
	% 	\item RMSProp
	% \end{itemize}
	% \textbf{Regularisierung} Methoden um Overfitting vorzubeugen
	% \begin{itemize}
	% 	\item Rauschen auf Eingabedaten, Gewichten, Ausgabe (Zufallswerte hinzufügen)
	% 	\item Datenaugmentierung
	% 	\item Early-Stopping: Laufende Validierung auf Validierungsset, Trainingsabbruch wenn Accuracy abnimmt
	% 	\item Dropout (siehe Layer)
		
	% \end{itemize}
	% \textbf{Pre-Training} Verwende Startgewichte, die bereits auf ähnlichen Daten trainiert wurden $\rightarrow$ besser als zufällige Initialisierung.\\
	% \textbf{Transferlearning} Verwende vortrainiertes Netz und trainiere nur einzelne Schichten (z.B. letzte Schicht für Klassifikation) neu.

	% \subsection{Netzarchitektur}
	% \textbf{Perceptron} (= künstliches Neuron) stellt lineare Trennung (binäre Klassifikation) dar. Kann Funktionen AND, OR und NOT lernen, nicht aber XOR.\\
	% \includegraphics[width=\linewidth]{figures/perceptron.png}\\
	% mit Inputs $a_j$ gewichtet mit $W_{ji}$ (ergeben zusammen die Gewichtsmatrix $W$), addiert mit Bias $b$, Outputs $a_i$ und der Aktivierungsfunktion $g$ mit der der Output berechnet wird.
	% $$Ausgabe_{Schicht(x)} = g(Eingabe_{Schicht(x)} \cdot W + b) = Eingabe_{Schicht(x+1)}$$
	% \textbf{Aktivierungsfunktion} muss bei Multi-Layer-Perceptronen (MLP) nicht-linear sein, da sonst nicht mehr Information gespeichert werden kann.
	% $$(x \cdot W + b) \cdot V + a = x \cdot W \cdot V + b \cdot V + a = x \cdot W' + b'$$

	% \section{Layer}
	% Mögliche Schichten aus denen ein neuronales Netz bestehen kann.
	% \subsection{Fully Connected / Dense}
	% Voll verbundene Schicht, d.h. jeder Input landet in jedem Neuron. Wird z.B. als letzter Layer zur Klassifikation verwendet, um Vektor mit Logits für Klassen auszugeben.\\
	% \textbf{Parameter} $X \in \R^{1 \times a}$: Eingabe, $W \in \R^{a \times n}$: Gewichtsmatrix, $b \in \R^{1 \times n}$: Bias, $n$: Anzahl der Neuronen, $Y \in \R^{1 \times n}$: Ausgabe\\
	% \textbf{Forward-Pass} $$Y = X \cdot W + b$$
	% \textbf{Backward-Pass} $$\delta X = \delta Y \cdot W^T$$
	% \textbf{Calculate Delta Weights} $$\frac{\delta L}{\delta W} = X^T \cdot \delta Y$$ $$\frac{\delta L}{\delta b} = \delta Y$$
	% \subsection{Aktivierungsfunktion}
	% Elementweise Anwendung\\
	% \begin{tabular}{|l|l|l|}
	% \hline
	% \textbf{Funktion} & \textbf{Forward} & \textbf{Backward}\\
	% \hline
	% tanh & $Y = \tanh(X)$ & $\delta X = (1 - \tanh^2(X)) \odot \delta Y$\\
	% \hline
	% Sigmoid & $Y = \sigma(X) = \frac{1}{1 + e^{-X}}$ & $\delta X = (\sigma(X) \cdot (1-\sigma(x))) \odot \delta Y$\\
	% \hline
	% ReLU & $Y = ReLU(X)$ & $\delta X = ReLU'(X) \odot Y$\\
	% \hline
	% \end{tabular}\\
	% \\
	% ReLU: $ReLU(x) = x > 0 ? x : 0$, $ReLU'(x) = x > 0 ? 1 : 0$
	% \subsection{Softmax}
	% Normalisiert eine Menge von Werten, sodass deren Summe 1 ergibt. Wird zur Berechnung der Wahrscheinlichkeitsverteilung bei Klassifikation verwendet.\\
	% \textbf{Forward-Pass} $$Y = softmax(X) = \frac{e^{x_i}}{\sum_i e^{x_i}}$$
	% \textbf{Backward-Pass} $$\delta X = \delta Y \cdot
	% \begin{bmatrix}
	% \frac{\delta x_1}{\delta y_1} & \cdots & \frac{\delta x_n}{\delta y_1}\\
	% \cdots & \ddots & \cdots\\
	% \frac{\delta x_1}{\delta y_n} & \cdots & \frac{\delta x_n}{\delta y_n}
	% \end{bmatrix}
	% $$
	% \subsection{Dropout}
	% Für Regularisierung verwendet. Ein Teil der Gewichte wird zufällig je Durchlauf auf 0 gesetzt (deaktiviert). Wird nicht trainiert, sollten alle Gewichte weitergegeben, aber durch die Dropout-Rate geteilt werden, da sonst eine zu hohe Aktivierung der nächsten Schicht statt findet.\\
	% \textbf{Parameter} $d$: Dropout-Rate gibt den prozentualen Anteil der zu deaktivierenden Gewichte an.
	% \subsection{Convolutional}
	% Extrahiert Features aus einer Matrix. Wird oft in der Bildklassifizierung verwendet oder NLP zur Satzklassifikation (Kernel Größe der Embeddings $\times$ Anzahl betrachteter Wörter).\\
	% \textbf{Parameter}\\
	% $X \in \R^{h \times w \times d}$: Eingabe\\
	% $(fh, fw)$: Filtergröße\\
	% $fn$: Anzahl Filter\\
	% $F \in \R^{fh \times fw \times fd \times fn}$: Filtertensor\\
	% $b \in \R^{fn}$: Bias (ein Wert pro Ausgabechannel, wird auf jeden Wert des Channels addiert)\\
	% $Y \in \R^{(h-fh+1) \times (w-fw+1) \times fn}$: Ausgabe\\
	% \textbf{Optionale Parameter} Stride: wie weit der Filter verschoben wird (default = 1), Dilation: zusätzliche 0en im Filter (Filter über nicht benachbarte Elemente)\\
	% \textbf{Padding} Covolution verkleinert die Daten. Um dies zu verhindern, kann die Eingabe gepadded werden (Hinzufügen von 0en am Rand der Eingabe).\\
	% \textit{Half-Padding}: Ausgabegröße = Eingabegröße, $ph = \lfloor \frac{fh}{2} \rfloor$, $pw = \lfloor \frac{fw}{2} \rfloor$\\
	% \textit{Full-Padding (Full-Convolution)}: Ausgabegröße $>$ Eingabegröße, $ph = fh - 1$, $pw = fw - 1$\\
	% \textbf{Forward-Pass} $*$ bezeichnet die Faltungsoperation\\
	% \includegraphics[width=\linewidth]{figures/convolution.png}
	% $$Y = X * F + b$$
	% \textbf{Backward-Pass} $*_F$ bezeichnet die Full-Convolution\\
	% \includegraphics[width=\linewidth]{figures/convolution-backward.png}
	% $$\delta X = \delta Y *_F rot^{180}_{h,w}(trans(F))$$
	% \textbf{Calculate Delta Weights} $*_{ch}$ bezeichnet die Channel-Wise-Convolution, $f$ ist ein Element des Filtertensors
	% $$\frac{\delta L}{\delta f} = X *_{ch} \delta Y$$
	% $$\frac{\delta L}{\delta b_f} = \sum_i \delta y_{i,f}$$
	% \subsection{Pooling}
	% Reduziert die Werte innerhalb eines Filters auf einen Wert (z.B. Maximum oder Durchschnitt). Filter wird ohne Überlappung (mit Stride) verschoben.\\
	% \textbf{Forward-Pass}\\
	% \includegraphics[width=\linewidth]{figures/max-pooling.png}\\
	% \textbf{Backward-Pass} Nur die Elemente, die im Forward-Pass an der Berechnung des Outputs beteiligt waren, bekommen anteilig oder ganz das Delta des jeweiligen Outputs.
	% $$\delta x_i = (x_i == max) ? \delta y : 0$$
	% \subsection{Vanilla RNN}
	% Recurrent Neuronal Networks (RNNs) werden verwendet, um Daten unterschiedlicher Länge zu verarbeiten. Dabei wird ein Hidden State $h$ in jedem Schritt um Informationen der Eingabe ergänzt.\\
	% \textbf{Parameter}\\
	% $[x_0, \cdots, x_n] = X \in \R^{n \times m}$: Eingabevektoren mit jeweils Länge $m$\\
	% $g$: Zwischenvariable nach der Addition vor $\tanh$\\
	% $h$: Hidden State ($h_{-1}$ kann mit Nullen initialisiert werden)\\
	% $W_x, W_h, W_o$: Gewichtsmatrizen\\
	% $b$: Bias\\
	% $[o_0, \cdots, o_n] = O \in \R^{n \times l}$: Ausgabevektoren mit jeweils Länge $l$\\
	% \textbf{Funktionsweise}\\
	% \includegraphics[width=\linewidth]{figures/vanilla-rnn.png}\\
	% \textbf{Forward-Pass}
	% $$h_t = \tanh(h_{t-1} \cdot W_h + x_t \cdot W_x + b)$$
	% $$o_t = h_t \cdot W_o$$
	% \textbf{Backward-Pass}
	% $$\delta h_t = \delta o_t \cdot W_o^T + \delta g_{t+1} \cdot W_h^T$$
	% $$\delta g_t = \delta h_t \cdot (1 - \tanh^2(g_t)) = \delta h_t \cdot (1 - h_t^2)$$
	% $$\delta x_t = \delta g_t \cdot W_x^T$$
	% \textbf{Calculate Delta Weights}
	% $$\delta W_h = \sum_t h_{t-1}^T \cdot \delta g_t$$
	% $$\delta W_x = \sum_t x_t^T \cdot \delta g_t$$
	% $$\delta W_o = \sum_t h_t^T \cdot \delta o_t$$
	% $$\delta b = \sum_t \delta g_t$$
	% \subsection{Gru}
	% Verwendet zwei Gates (wie $g$ im Vanilla RNN), um zu bestimmen, was aus dem alten Internal State $h$ übernommen wird und was von den aktuell verarbeiteten Daten hinzugefügt wird.\\
	% \textbf{Funktionsweise}\\
	% \includegraphics[width=\linewidth]{figures/gru.png}
	% \subsection{LSTM}
	% Verwendet drei Gates (Forget, Input, Output) und zusätzlichen State $c$\\
	% \textbf{Funktionsweise}\\
	% \includegraphics[width=\linewidth]{figures/lstm.png}
	% \subsection{Highway Layer}
	% Verwendet ein Gate, das bestimmt, ob der Layer wie ein Fully Connected Layer funktioniert oder lediglich den Input weitergibt. Wird verwendet bei sehr tiefen Netzwerken ($>$ 100 Layer), um Daten über lange Zeit/viele Layer hinweg nutzbar zu machen.\\
	% \textbf{Forward-Pass}
	% $$Y = (X \cdot W + b) \cdot T(x) + x \cdot (1 - T(x))$$
	% $$T(x) = \sigma(x \cdot W_T + b_T)$$

	% \section{Loss-Funktionen}
	% $X$: Eingabe der Loss-Funktion bzw. Ausgabe des Netzes (Prediction)\\
	% $T$: Erwartetes Ergebnis (Ground Truth)
	% \subsection{Cross Entropy}
	% Negative-Log-Likelihood, Cross-Entropy\\
	% \textbf{Forward-Pass} $$L = -\sum_i t_i \cdot \log(x_i)$$
	% \textbf{Backward-Pass} $$\frac{\delta L}{\delta x_i} = -\frac{t_i}{x_i}$$
	% \subsection{Mean Squared Error}
	% Euklidischer Loss, Mean-Squared-Error, $l_2$-Loss\\
	% \textbf{Forward-Pass} $$L = \sum_i \frac{1}{2} (x_i - t_i)^2$$
	% \textbf{Backward-Pass} $$\frac{\delta L}{\delta x_i} = t_i - x_i$$

	% \section{Bildverarbeitung}
	% \includegraphics[width=\linewidth]{figures/bildverarbeitung.png}
	% \subsection{Klassifizierung}
	% In der Regel neuronales Netz mit Convolution und Fully Connected Layern, das das gesamte Bild als Eingabe bekommt.
	% \subsection{Segmentierung}
	% Zuweisung einer Klasse für jeden Pixel eines Bildes.\\
	% \textbf{Sliding Window} Klassifizere kleinere Bildausschnitte. Probleme: Keine Flächen (benachbarte Vorhersagen haben keinen direkten Einfluss), Auflösung wird niedriger durch CNN, kleiner Stride führt zu Mehrfachberechnung\\
	% \textbf{Fully Convolution Network} CNN als Encoder, das Größe verkleinert. Anschließend hochskalieren (Decoder) $\rightarrow$ schnell aber ungenau\\
	% \textbf{Skip Connections} Addieren/Konkatinieren mit Werten aus vorherigen Layern\\
	% \includegraphics[width=\linewidth]{figures/skip-connections.png}\\
	% \textbf{Transposed Convolution} Anwenden der transponierten Convolution zur Bildvergrößerung (wie im Backward-Pass der regulären Convolution)\\
	% \includegraphics[width=\linewidth]{figures/transposed-convolution.png}\\
	% \textbf{Unpooling} Umkehren des Poolings analog zum Backward-Pass (z.B. bei Max-Pooling bekommt das ursprüngliche Maximum alles). Kann nur mit einer Pooling-Schicht verwendet werden, die umgekehrt werden soll $\rightarrow$ Alternative zum Upsampling
	% \subsection{Objekterkennung}
	% \textbf{Single Shot Detector} Definiere Standard Bounding Boxes (BB) im Bild. Netwerk soll Klasse und Abweichung von einer Standard BB angeben.
	% $$L = \frac{1}{N} (L_{conf} + \alpha L_{loc})$$
	% mit $L_{conf}$ als Cross-Entropy über die Klassen und $L_{loc} = - \sum_n^N smooth_{l_1}(l_n - g_n)$\\
	% \includegraphics[width=\linewidth]{figures/single-shot-detector.png}\\
	% \textbf{Multi Object Detection} mehrere BBs (ca. 4) pro Pixel. BB gilt als gematched, wenn Flächeninhalt der Vorhersage mit BB zu z.B. 50 Prozent übereinstimmt.\\
	% \textbf{Non-Maximum Suppression} Da oft mehrere BBs matchen, suche BB mit höchster Konfidenz und lösche alle BBs, die z.B. mind. 50 Prozent Overlapping haben. Wiederhole, bis keine BBs mehr gelöscht wurden.

	% \section{Sequence2Sequence}
	% Abbildung von Sequenzen auf Sequenzen, z.B. bei OCR (Eingabe: Pixelspalten, Ausgabe: Buchstabenklassifikation) oder Spracherkennung. Kann z.B. mit LSTMs umgesetzt werden.\\
	% \textbf{Greedy-Decoder} Ermittelt Klasse durch Maximum auf Ausgabe $\rightarrow$ Entfernen doppelter Zeichen $\rightarrow$ Blanks entfernen\\
	% \textbf{Labellings} Beschreibt die Abbilung der Netzausgabe auf sinnvolle Sequenz.
	% $$B: L'^{T} \rightarrow L^{\leq T}$$
	% $$B(a-ab-) = B(-aa--aabb) = aab$$
	% $$B^{-1}(aab) = {a-ab-, ..., aa-ab, ...}$$
	% Summiere die Wahrscheinlichkeiten aller möglichen Pfade, die zum gewünschten Label führen.\\
	% \textbf{CTC-Loss (Connectionist Temporal Classification)}\\
	% Wird angewendet, wenn die Eingabe eine Zeitsequenz der Länge $T$ ist und als Ausgabe eine unalignierte Labelsequenz der Länge $\leq T$ vorliegt\\
	% \includegraphics[width=\linewidth]{figures/ctc-algorithmus.png}
	% $$P(l|x) = \sum_{\pi \in B^{-1}(l)} P(\pi|x)$$
	% mit $l$ als gewünschte Sequenz und $x$ als Netzausgabe\\
	% \textit{Loss}: Maximiere $P(l|x)$ bzw. minimiere $-\log(P(l|x))$\\	
\end{document}
